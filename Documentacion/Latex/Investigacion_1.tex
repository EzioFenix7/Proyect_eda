\documentclass{article}
\usepackage[utf8]{inputenc}
\usepackage[spanish]{babel}
\usepackage{graphicx}
\usepackage{hyperref}
\graphicspath{{imagenes}}
\title{Manual del proyecto 1 de la materia Estructuras de Datos 2}
\author{Barrera Peña Víctor Miguel \\
		Roldán Franco Luis Miguel \\
		Velázquez de León Lavarrios Alvar 
	 }
\date{Entrega 25/03/2019}

%\setlength{\parskip}{1em}

\begin{document}
	\maketitle
	\section{Introducción}
	\section{Marco Teórico}
		\paragraph{Algoritmo de Mezcla equilibrada}
		\paragraph{Algoritmo Radix}
		\subsection{Algortimo de Polifase}
		
		\section{title}
	\subsection{Análisis}
	\subsection{Diseño}
	\section{Desarrollo}
		
	\subsection{Implementación}
		Primero es necesario leer la documentación de los algoritmos que se implementaron para ordenar  para comprender como es que se dividió el programa para su posterior explicación general.
		
	
	\paragraph*{Explicación del menu principal}
	
	\paragraph*{Explicación de Polifase}
	
	\paragraph*{Explicación Radix Sort}
	
	\paragraph*{Explicación de Mezcla equilibrada}
	
	

	
	\subsection{Pruebas}
		\paragraph*{Pruebas previas de código}
		En esta sección se adicionan las pruebas previas del código que se fue implementando en el código final.
		
			\begin{itemize}
				\item \url{https://github.com/EzioFenix7/Proyect_eda/tree/master/pruebas\%20proyecto}
				\href{https://github.com/EzioFenix7/Proyect_eda/tree/master/pruebas\%20proyecto}{ Pruebas de Snippets de código del que se basa el proyecto final}
			\end{itemize}
		
		\paragraph*{Pruebas de la versión final del proyecto}
		En esta sección se muestra el programa funcionando en una computadora con sistema Windows, para probar que el proyecto funcione, no se ha dicho que no funcione en otro sistema operativo, pero para caso de análisis lo hacemos solo en este.
	\subsection{Documentación}
		\begin{itemize}
			\item \url{https://github.com/EzioFenix7/Proyect_eda/tree/master/Documentacion/Polifase}
			\href{https://github.com/EzioFenix7/Proyect_eda/tree/master/Documentacion/Polifase}{Direccion de documentacion de polifase}
			
			\item \url{https://github.com/EzioFenix7/Proyect_eda/tree/master/Documentacion/Radix}
			\href{https://github.com/EzioFenix7/Proyect_eda/tree/master/Documentacion/Radix}{Direccion de documentacion de Radix}
			
			\item 	\url{https://github.com/EzioFenix7/Proyect_eda/tree/master/Documentacion/Equilibrada}
			\href{https://github.com/EzioFenix7/Proyect_eda/tree/master/Documentacion/Equilibrada}{Direccion de documentacion de Mezcla Equilbrada}
		\end{itemize}
	\subsection{Mantenimiento}
	\section{Conclusión}
	\subsection{Observaciones}
\end{document}
