\documentclass[12pt]{report}
\usepackage[utf8]{inputenc}
\usepackage[spanish]{babel}
\usepackage{graphicx}
\graphicspath{{imagenes}}
\title{Manual del proyecto I \\ Materia: \\Estructuras de Datos y Algoritmos II\\ 
\large Métodos de ordenamiento}

\author{
	Equipo:\\
	Barrera Peña Víctor Miguel \\
	Roldán Franco Luis Miguel \\
	Velázquez de León Lavarrios Alvar 
}
\date{Entrega 25/03/2019}

\begin{document}
	\maketitle
	\chapter{Primeros pasos}
	Este programa  esta hecho para poder ordenar un archivo mediante 3 métodos de ordenamiento como son: Polifase, Radix, Mezcla equilibrada.
	Para poder lograr el perfecto funcionamiento de este programa se tiene que considerar lo siguiente:
	\begin{itemize}
		\item Si el programa especifica que se introduzca un número para seleccionar una opcion, no es valido introducir un carácter o signos diferentes a números, en caso de introducir estos, sucederá un error y tendrá que volverse a iniciar el programa.
		\item No debe estar abierto durante la ejecución el archivo ``texto.txt'' , ya algunos editores de texto restringe la modificación de este durante la ejecución.
		\item Los archivos auxiliares generados por Polifase deben de ser eliminados manualmente cuando se desee realizar otra ejecución.
		Esto con el propósito de liberar espacio en el escritorio.
		\item Si un método especifica que se introduzcan datos estos deben de ser correcto, o de lo contrario podrían surgir errores.
	\end{itemize}
	Para el archivo tiene a ordenar debe cumplir con ciertos requisitos para poder ser ordenado que se en listan a continuación:
	\begin{itemize}
		\item Para Radix y Mezcla Equilibrada tiene que llamarse ``texto.txt'' y tiene que estar en el escritorio de la computadora desde donde se ejecuta.
		\item Para Polifase existen dos archivos ``textop1.txt" y ``textop2.txt" y también se deben de ubicar en el escritorio de la computadora.
		\item Los archivos deben de contener cadenas de tres caracteres separadas por comas. Por ejemplo: {uad,eed,osr,...}.
	\end{itemize}
	
	\chapter{Paso 1 ¿Qué desea hacer? (Menú principal)}
	El menú principal al inicio sólo te muestra 2 opciones ``Salir'' o ``ordenar'' Seleccione alguno de los 2 introduciendo un número.
	\begin{itemize}
		\item Si se introduce `0' terminará la ejecución del programa.
		\item Si introduce 1 , podrá acceder al menú donde seleccionará un método de ordenamiento (ver la sección de subsumen del manual de usuario).
		\item Si se introduce otro número se le volverá a pedir un número hasta que escoja una de las 2 anteriores opciones
	\end{itemize}

	\chapter{Paso 2 Seleccione un método de ordenamiento (Menú de ordenamiento)}
	Debe considerar que tiene que existir un archivo que desee emplear para ordenar, debe estar en el escritorio de su computadora.\\
	
	Aparecerá un menú donde usted podrá seleccionar con que método desea ordenar el archivo, introducirá un numero entre el 0 y el 3 de acuerdo al método de ordenamiento que desee.
	Si introduce un número fuera de esta lista de valores, el programa igual que en el caso anterior, le preguntará por un número valido entre 0 y 3.\\
	
	Si se seleccionó Polifase se deben ingresar desde teclado el nombre del archivo especificado para este método y el número de llaves que se deseen leer. Durante su ejecución se
	generarán varios archivos auxiliares que sirven como parte de las iteraciones que realiza el programa. Los archivos auxiliares de Polifase tienen por nombre $PolifaseX.txt$ y se deben de
	abrir con algún editor de textos para ver las iteraciones realizadas por el programa. El último archivo generado por el programa es el archivo final ordenado.

	\chapter{Paso 3 Revisar el archivo ya ordenado}
	\paragraph{Radix y Mezcla Equilibrada}
	En el escritorio de la computadora, el archivo final saldrá con el mismo nombre ``texto.txt'' el cual ya estará ordenado.\\
	
	Podrá realizar otro ordenamiento, pero si lo realiza inmediatamente no sucederá nada, porqué ya esta ordenado. Lo mejor sería  darle `0' para que termine la ejecución y de esta manera revisarlo con más detenimiento.
	
	\paragraph{Polifase}
	En el escritorio de la computadora se encontraran varios archivos auxiliares con la nomenclatura $PolifaseX.txt$, pero solamente el último que se generó es el archivo
	que ya se encuentra ordenado.\\
	
	Para realizar algún otro ordenamiento se deben \textbf{eliminar todos los archivos auxiliares} generados de manera manual, porque de lo contrario afectaría ejecuciones
	futuras de los otros métodos y del mismo.\\
	
	Con lo anterior dichos es posible decir que se termino con éxito el programa, si se presento un error es porque no se siguió las instrucciones de este manual,
	o en su defecto hay un error que no esta asociado con este. 
	En caso contrario Contactar con los creadores de este  programa.
\end{document}
