\documentclass[12pt]{report}
\usepackage[utf8]{inputenc}
\usepackage[spanish]{babel}
\usepackage{graphicx}
\graphicspath{{imagenes}}
\title{Manual del proyecto I \\ Materia: \\Estructuras de Datos 2\\ 
\large Métodos de ordenamiento}

\author{
	Equipo:\\
	Barrera Peña Víctor Miguel \\
	Roldán Franco Luis Miguel \\
	Velázquez de León Lavarrios Alvar 
}
\date{Entrega 25/03/2019}

\begin{document}
	\maketitle
	\chapter{Primeros pasos}
	Este programa  esta hecho para poder ordenar un archivo mediante 3 métodos de ordenamiento como son: Polifase, Radix, Mezcla equilibrada.
	Para poder lograr el perfecto funcionamiento de este programa se tiene que considerar lo siguiente:
	\begin{itemize}
		\item No se debe borrar ningún archivo del programa, con esto quiero decir que si se genera un nuevo documento el programa, no se debe borrar, este automáticamente se borrará.
		\item Si el programa especifica que se introduzca un número para seleccionar una opcion, no es valido introducir un carácter o signos diferentes a números, en caso de introducir estos, sucederá un error y tendrá que volverse a iniciar el programa.
		\item No debe estar abierto durante la ejecución el archivo ``texto.txt'' , ya algunos editores de texto restringe la modificación de este durante la ejecución.
	\end{itemize}
	Para el archivo tiene a ordenar debe cumplir con ciertos requisitos para poder ser ordenado que se en listan a continuación:
	\begin{itemize}
		\item Tiene que llamarse ``texto.txt'' y tiene que estar en el escritorio de la computadora desde donde se ejecuta.
		\item Dentro del contenido  archivo, debe empezar por un número.
		\item La separación entre  números debe estar unicamente dada por una coma(,) y para delimitar un número (su final) también debe tener una. Por ejemplo el archivo contendría en su interior la siguiente cadena={3,9,6,}
		\item El ordinal de números no esta determinado , es al gusto. 
		\item En el archivo original no debe contener saltos de linea, en caso de contener, este debe empezar de nuevo con un número, y el anterior debe renglon debe haber terminado en coma.
	\end{itemize}
	
	\chapter{Paso 1 ¿Qué desea hacer? (Menú principal)}
	El menú principal al inicio sólo te muestra 2 opciones ``Salir'' o ``ordenar'' Seleccione alguno de los 2 introduciendo un número.
	\begin{itemize}
		\item Si se introduce `0' terminará la ejecución del programa.
		\item Si introduce 1 , podrá acceder al menú donde seleccionará un método de ordenamiento (ver la sección de subsumen del manual de usuario).
		\item Si se introduce otro número se le volverá a pedir un número hasta que escoja una de las 2 anteriores opciones
	\end{itemize}

	\chapter{Paso 2 Seleccione un método de ordenamiento (Menú de ordenamiento)}
	Debe considerar que tiene que existir un archivo que desee emplear para ordenar, debe estar en el escritorio de su computadora.
	
	Aparecerá un menú donde usted podrá seleccionar con que método desea ordenar el archivo, introducirá un numero entre el 0 y el 3 de acuerdo al método de ordenamiento que desee.
	Si introduce un número fuera de esta lista de valores, el programa igual que en el caso anterior, le preguntará por un número valido entre 0 y 3.

	\chapter{Paso 3 Revisar el archivo ya ordenado}
	En el escritorio de tu computadora, saldrá con el mismo nombre el archivo ``texto.txt'' y ya estará ordenado.
	\\Podrá realizar otro ordenamiento, pero si lo realiza inmediatamente no sucederá nada, porqué ya esta ordenado. Lo mejor sería  darle `0' para que termine la ejecución y de esta manera revisarlo con más detenimiento.
	
	Con lo anterior dichos es posible decir que se termino con éxito el programa, si se presento un error es porque no se siguió las instrucciones de este manual,
	o en su defecto hay un error que no esta asociado con este. 
	En caso contrario Contactar con los creadores de este  programa.
\end{document}